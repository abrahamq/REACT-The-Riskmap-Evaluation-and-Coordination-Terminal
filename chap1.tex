%% two pager here's the problem here's how its structured
\chapter{Introduction} Natural disasters are a constant threat to societies all
over the planet. Among natural disasters, flooding is the most common calamity
in the world~\cite{chanFloodRiskAsia2012}.  Flood related deaths account for
half of all deaths from natural disasters~\cite{ohlFloodingHumanHealth2000}.
Although flooding impacts both developed and developing countries, developing
nations face much higher mortality rates as a result of flooding since they lack
resources to adequately mitigate
hazards~\cite{quarantelliUrbanVulnerabilityDisasters2003,
ahernGlobalHealthImpacts2005}. Deltaic megacities in
developing countries are particularly at risk because unregulated
urbanization, rising population and climate change are increasing the rate at
which floods occur~\cite{chanFloodRiskAsia2012}. Moreover, there is little data
that is available before, during, and after a disaster to help stakeholders
mitigate hazards~\cite{meierDigitalHumanitariansHow2015}.
% Data scarcity makes it
% hard to pinpoint where to direct aid during disasters and where to make
% infrastructure improvements after
% disasters~\cite{ranaMultidimensionalModelVulnerability2018}.

%% TODO: this is pretty disjointed
Various stakeholders, such as humanitarian NGOs, government emergency
responders and affected citizens, have
different but overlapping interests with regards to disaster management.
Government and NGOs work together to provide relief and mitigate damages from
flooding~\cite{chanResilientFloodRisk2018}, while citizens
look for relevant information and try to reduce their
risk by avoiding heavily affected
areas~\cite{viewegMicrobloggingTwoNatural2010}. Information is at the core of
disaster management; however, data scarcity makes it hard for emergency
personnel to optimize their use of resources, while citizens have an abundance
of information about their surroundings but must be careful not to trust
incorrect or outdated information about broader
areas~\cite{quarantelliProblematicalAspectsInformation1997}.

Disaster information systems can connect affected communities with
Emergency Operations Centers (EOCs), thereby bridging the information gap
between responders and citizens. Many such disaster information systems have
been developed, but they often suffer from a lack of institutional
buy-in~\cite{aminDataNaturalDisasters2008}. The lack of engagement can be partly
attributed to the difficulty of adding data gathering responsibilities to
emergency personnel that have little time during crises. Asking citizens to
submit information is a solution to this problem; however, crowdsourcing brings
its own issue: in times of crisis EOCs can suffer from information overload when
they are presented with too much
data~\cite{tierneyFacingUnexpectedDisaster2001}.

The REACT system uses novel machine learning and human computer interaction
research to reduce information overload from crowd sourced data in EOCs, thereby
decreasing disaster response time. REACT classifies reports as indicating heavy
flooding or not through an ensemble model. It extracts key features from
each of the parts of a report (text, picture, metadata) using domain specific
techniques and then uses a small dense neural net to classify the citizen
report.

First we establish the motivation for using citizens as sensors and
analyzing this noisy data using machine learning. We then review different
machine learning techniques that have been used in crisis information
systems, including those that also utilize social media.  Finally a novel
ensemble learning model is presented that can accurately predict large urban
flood events from crowdsourced data.
