%% This is an example first chapter.  You should put chapter/appendix that you
%% write into a separate file, and add a line \include{yourfilename} to
%% main.tex, where `yourfilename.tex' is the name of the chapter/appendix file.
%% You can process specific files by typing their names in at the 
%% \files=
%% prompt when you run the file main.tex through LaTeX.
\chapter{Introduction} Flooding is the most common natural disaster in the
world~\cite{chanFloodRiskAsia2012}.  Flood related deaths account for half of
all deaths from natural disasters~\cite{ohlFloodingHumanHealth2000}. Although
flooding impacts both developed and developing countries, developing nations
face much worse consequences as a result of flooding since they lack resources
to adequately mitigate
hazards~\cite{quarantelliUrbanVulnerabilityDisasters2003}. Deltaic megacities
in developing countries are particularly at risk because of unregulated
urbanization, rising population and climate change are increasing the rate at
which floods occur~\cite{chanFloodRiskAsia2012}. In
addition to the increase in disaster risk, there is little data that is
available before, during, and after a disaster to help stakeholders mitigate
hazards~\cite{meierDigitalHumanitariansHow2015}. Data scarcity makes it hard to
pinpoint where to direct aid during disasters and where to make infrastructure
improvements after disasters~\cite{ranaMultidimensionalModelVulnerability2018}.

%% TODO: this is pretty disjointed
Various stakeholders have different but overlapping interests with regards to
disaster management. Government and NGOs work together to provide relief and
mitigate damages from flooding~\cite{chanResilientFloodRisk2018}, while Citizens
look for relevant flood information and try to reduce their
risk~\cite{viewegMicrobloggingTwoNatural2010}. Information is at the core of
this interaction; however, data scarcity makes it hard for emergency personnel
to optimize their use of resources, while citizens have an abundance of
information about their surroundings but must be careful not to trust incorrect
or outdated information about broader
areas~\cite{quarantelliProblematicalAspectsInformation1997}. The natural
solution is for citizens on social media to submit real time reports to the
Emergency Operations Center (EOC), which is tasked with using those reports to
inform citizens.  There is one problem with this solution, in times of crisis
EOCs can suffer from information overload when they are presented with too much
information~\cite{tierneyFacingUnexpectedDisaster2001}.

The REACT system uses novel machine learning and human computer interaction
research to reduce information overload in EOCs, thereby decreasing disaster
response time. REACT classifies reports as indicating heavy flooding or not
through an ensemble model. It first extracts key features from each of the parts
of a report (text, picture, metadata) using known techniques and then uses a
small dense neural net to classify the citizen report.

%% TODO I feel like I'm restating myself here?
\section{Emerging Risk In South and Southeast Asia} Global climate change is
`expected to increase the frequency and intensity of
floods'~\cite{ahernGlobalHealthImpacts2005}.  Urban areas are particularly at
risk from flooding; unchecked development and rising population have created
megacities that regularly experience flooding~\cite{chanFloodRiskAsia2012}.
Nowhere is this more apparent than in South and Southeast Asia, where the
severity of floods has been increasing over the past several
decades~\cite{tortiFloodsSoutheastAsia2012}.

Of the world's 33 megacities, over 60 percent are located in developing Asian
Countries~\cite{unitednationsdepartmentofeconomicandsocialaffairsWorldCities20162016}.
These cities face a looming crisis as flood risk increases, but there are also
unique opportunities for risk mitigation. Megacities are characterized by high
population density. This high density leads to an increase in economic damages
and loss of life, but it also means that there are large numbers of citizens
that have disaster information they would like to share with
others~\cite{chanFloodRiskAsia2012}.

\section{History of Disaster Informatics} One of the best known and earliest
work in mapping disasters was John Snow's use of maps to find the
source of the 1857 Cholera outbreak in London\cite{rogersJohnSnowData2013}. This
example is generally taught to all new students of epidemiology and illustrates
the need not only for up to date information, but for systems that ease the
analysis of this information. In John Snow's case, the map was the tool that
allowed him to visualize the spread of the disease and effectively take action
that ended the outbreak.

In more recent times, the need for Information Technology (IT) in disaster
management has been clear since the mid 1980s when computers became user
friendly enough to be used during
disasters~\cite{universityTerminalDisastersComputer1986}. Now many EOCs use
Geographic Information Systems (GIS), inventory control systems, and online
messaging systems among other technology in order to organize spatial data and
analyze disaster information. While some systems have helped some regions to
better respond to disasters, researchers have often stated that `there
are many reasons to remain skeptical about the idea that technology will provide
a panacea for emergency management problems'~\cite{tzemosUseGISFederal1995,
tierneyFacingUnexpectedDisaster2001, perryNaturalDisasterManagement2007}. A
number of potential negative effects associated with disaster management
technology have been identified: the potential for the technology to increase
social inequality, the potential of information overload, and the dissemination
of incorrect and outdated information~\cite{quarantelliProblematicalAspectsInformation1997,
flentgeDesigningContextAwareHCI}.

\subsection{Social Media and Disasters}
%% TODO is this in the right place??
The history of social media and the
hashtag is invariably linked to disaster communication.  It was during the San
Diego bush fires of 2007 that the hashtag was first widely used on
twitter~\cite{salazarHashtagsAnnotatedHistory2017}.

Much work has been done in passively listening to social media streams
in order to better understand how disasters unfold and how humans use
social media as a communication tool during disaster events. Many of
these studies use hand labeled tweets in order to analyze which
percentage of them 

Digital Humanitarians use twitter to help spatially locate needs in
Haiti after the 2008 earthquake~\cite{meierDigitalHumanitariansHow2015}

Twitter tale of 3 hurricanes~\cite{alamTwitterTaleThree2018}

Quarantelli emphaszied that  `management of hazards is fundamentally social in
nature and not something that can be achieved strictly through technological
upgrading'~\cite{tierneyFacingUnexpectedDisaster2001} yet social media brings
human behavior into a machine readable format that can be used to provide
further information during disasters.

\subsection{Crowdsourcing vs. Passive Listening}
\begin{quote}
	\textit{Since humanitarian organizations
	don't ask eyewitnesses on social media to report information on needs
	and impact groups like the Red Cross have to rely on witnesses sharing
	relevant information by chance.}
		\begin{flushright}
		--- Patrick Meier~\cite{meierDigitalHumanitariansHow2015}
		\end{flushright}
\end{quote}

As Patrick Meier points out in \textit{Digital Humanitarians}, passively
listening to twitter data streams and hoping that someone posts relevant
disaster information is not always a winning strategy. One solution to this
problem is to have paid workers that collect information and enter it into
disaster information systems as in~\cite{aminDataNaturalDisasters2008}, which
details how paid workers were used to input data from citizens during the
Mozambique floods of 2007; however, this method is expensive and does not scale
well. The same report states that `data processing and consolidation [were]
difficult' and that `the few data entry clerks struggled to keep
up'~\cite{aminDataNaturalDisasters2008}.


Citizens as sensors Geosocial
intelligence Holderness~\cite{holdernessSocialMediaGeoSocial2015a}.

%% TODO not sure where to place this section?
\section{The Riskmap System}\label{chap1:riskmap}

	\subsection{Need for open data}
	Creating bespoke information systems at the beginning of disasters has been the
	norm~\cite{aminDataNaturalDisasters2008}; however this
	means that disaster response organizations must become acclimated with the
	system at the same time that they are dealing with disaster situations. 
	
	Researchers have shown the need to create open sourced crowdsourced emergency systems
	that provide open data~\cite{avvenutiNeedOpeningCrowdsourced2018a}. The
	Riskmap system was created to fill that need.

	\subsection{System Overview}
	The Riskmap system alleviates the load on emergency managers by
	centralizing reports from many social media sources. It also makes it
	easy not only for reports to come into the response center, but also for
	emergency managers to indicate which areas of a city are most affected
	at any one time. The data gathered during an event is persistent and
	available under the Creative Commons license, which allows researchers
	to track the flood over time and pinpoint areas that are particularly
	vulnerable to flooding, thus fulfilling the need for open data~\cite{holdernessSocialMediaGeoSocial2015a}.

	Riskmap consists of many different social media bots that are actively
	filtering social media streams and looking for citizens that might be
	reporting flooding events, it then reaches out to those users and asks
	them to submit a flood report that consists of a GPS location, the
	estimated flood height at that location, a picture, and a textual
	description. These reports are then displayed on a public map for other
	citizens to inform themselves. Furthermore, EOC personnel are able to
	access the Risk Evaluation Matrix (REM), a special dashboard that allows
	them to give even more information to citizens.
	
	The system has been in place in Jakarta and Chennai since 2016, and has seen
	hundreds of thousands of views during flood events.

\section{Conquering Information Overload}
	It is not enough to create an advanced system for consuming citizen
	reports, it is also necessary to ensure that this system does not
	consume resources that are already scarce during a disaster event, for
	example the time of emergency
	workers~\cite{aminDataNaturalDisasters2008}. It is also important to
	reduce the amount of time needed to create insights because if analyzing
	data takes too much time, then decision makers will make decisions
	without having fully analyzed the
	data~\cite{quarantelliUrbanVulnerabilityDisasters2003}.

	Using computers to automatically make sense of disaster data has long
	been a goal in disaster informatics, but only recently have machine
	learning techniques become good enough to be implemented in production
	emergency systems~\cite{meierDigitalHumanitariansHow2015}. Image
	recognition algorithms can provide summaries of objects and scenes found
	in user submitted photos~\cite{nguyenRapidClassificationCrisisRelated,
	donahueDeCAFDeepConvolutional2013}. Natural language processing can
	estimate the probability that a textual document is overall negative or
	positive and thereby give EOCs a shorthand way to summarize thousands of
	reports in short amounts of
	time~\cite{nguyenRapidClassificationCrisisRelated,
	nagyCrowdSentimentDetection2012}. Finally, ensemble learning methods can
	learn relationships between disparate datasets and synthesize a single
	result~\cite{mouzannarDamageIdentificationSocial2018}.
