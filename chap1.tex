%% two pager here's the problem here's how its structured
\chapter{Introduction} Flooding is the most common natural disaster in the
world~\cite{chanFloodRiskAsia2012}.  Flood related deaths account for half of
all deaths from natural disasters~\cite{ohlFloodingHumanHealth2000}. Although
flooding impacts both developed and developing countries, developing nations
face much worse consequences as a result of flooding since they lack resources
to adequately mitigate
hazards~\cite{quarantelliUrbanVulnerabilityDisasters2003}. Deltaic megacities
in developing countries are particularly at risk because of unregulated
urbanization, rising population and climate change are increasing the rate at
which floods occur~\cite{chanFloodRiskAsia2012}. In
addition to the increase in disaster risk, there is little data that is
available before, during, and after a disaster to help stakeholders mitigate
hazards~\cite{meierDigitalHumanitariansHow2015}. Data scarcity makes it hard to
pinpoint where to direct aid during disasters and where to make infrastructure
improvements after disasters~\cite{ranaMultidimensionalModelVulnerability2018}.

%% TODO: this is pretty disjointed
Various stakeholders have different but overlapping interests with regards to
disaster management. Government and NGOs work together to provide relief and
mitigate damages from flooding~\cite{chanResilientFloodRisk2018}, while Citizens
look for relevant flood information and try to reduce their
risk~\cite{viewegMicrobloggingTwoNatural2010}. Information is at the core of
this interaction; however, data scarcity makes it hard for emergency personnel
to optimize their use of resources, while citizens have an abundance of
information about their surroundings but must be careful not to trust incorrect
or outdated information about broader
areas~\cite{quarantelliProblematicalAspectsInformation1997}. The natural
solution is for citizens on social media to submit real time reports to the
Emergency Operations Center (EOC), which is tasked with using those reports to
inform citizens.  There is one problem with this solution, in times of crisis
EOCs can suffer from information overload when they are presented with too much
information~\cite{tierneyFacingUnexpectedDisaster2001}.

The REACT system uses novel machine learning and human computer interaction
research to reduce information overload in EOCs, thereby decreasing disaster
response time. REACT classifies reports as indicating heavy flooding or not
through an ensemble model. It first extracts key features from each of the parts
of a report (text, picture, metadata) using known techniques and then uses a
small dense neural net to classify the citizen report.

    First we establish the motivation for using citizens as sensors and
analyzing this noisy data using machine learning. We then review different
machine learning techniques that have been used in crisis information systems,
including those that also utilize social media.
Finally a novel ensemble learning model is presented that can accurately predict
large flood events from crowdsourced
data.
