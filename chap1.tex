%% This is an example first chapter.  You should put chapter/appendix that you
%% write into a separate file, and add a line \include{yourfilename} to
%% main.tex, where `yourfilename.tex' is the name of the chapter/appendix file.
%% You can process specific files by typing their names in at the 
%% \files=
%% prompt when you run the file main.tex through LaTeX.
\chapter{Introduction} Flooding is the most common natural disaster in the
world~\cite{chanFloodRiskAsia2012}.  Flood related deaths account for half of
all deaths from natural disasters~\cite{ohlFloodingHumanHealth2000}. 
Although flooding impacts both developed and
developing countries, developing nations face much worse consequences as a
result of flooding since they lack resources to adequately mitigate
disasters~\cite{quarantelliUrbanVulnerabilityDisasters2003}.
Unregulated urbanization, rising population and climate change all contrive to
increase the rate at which floods occur in developing megacities; furthermore,
there is little data about these disasters~\cite{chanResilientFloodRisk2018}.
Data scarcity makes it hard to
pinpoint where to direct aid during disasters and where to make infrastructure
improvements after disasters~\cite{ranaMultidimensionalModelVulnerability2018}.

Government and NGOs work together to mitigate damages from flooding.  Citizens
look for relevant flood information and try to reduce their risk.  Information
is at the core of this interaction; however, data scarcity makes it hard for
emergency personnel to optimize their use of resources, while citizens have an
abundance of information about their surroundings but must be careful not to
trust incorrect or outdated information about broader
areas~\cite{quarantelliProblematicalAspectsInformation1997}.  The natural
solution is for citizens on social media to submit real time reports to the
Emergency Operations Center (EOC), which is tasked with using those reports to inform
citizens.  There is one problem with this solution, in times of crisis EOCs can
suffer from information overload when they are presented with too much
information.

The REACT system uses novel machine learning and human computer interaction
research to reduce information overload in EOCs, thereby decreasing disaster
response time. REACT learns how Emergency Operations Centers (EOCs) classify the
severity of flood events given citizen submitted reports. REACT trains itself
through a gamified simulation of a disaster event. During a real disaster, REACT
digests social media reports and estimates how severely an event is impacting
different areas of a city and thereby helps EOCs to respond in the best manner
possible.

\section{History of Disaster Informatics}
Work in Mapping disasters
Epidemiology John Snow's use of maps to find the source of Cholera outbreak in
London\cite{rogersJohnSnowData2013}.

Technology can help disaster response; however, it also has the ability to cause
information overload\cite{tierneyFacingUnexpectedDisaster2001}



\section{The Riskmap System}\label{ch1:riskmap}


\subsection{Motivation for crowdsourced data}
Citizens as sensors
Geosocial intelligence
Holderness~\cite{holdernessSocialMediaGeoSocial2015a}~\ref{ch1:riskmap}.

Quarantelli emphaszied that  `management of hazards is fundamentally social in
nature and not something that can be achieved strictly through technological
upgrading'~\cite{tierneyFacingUnexpectedDisaster2001} yet social media brings
human behavior into a machine readable format that can be used to provide
further information during disasters.


\subsection{}

