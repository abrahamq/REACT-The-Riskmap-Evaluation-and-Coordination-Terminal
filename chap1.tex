%% This is an example first chapter.  You should put chapter/appendix that you
%% write into a separate file, and add a line \include{yourfilename} to
%% main.tex, where `yourfilename.tex' is the name of the chapter/appendix file.
%% You can process specific files by typing their names in at the 
%% \files=
%% prompt when you run the file main.tex through LaTeX.
\chapter{Introduction} Flooding is the most common natural disaster in the
world~\cite{chanFloodRiskAsia2012}.  Flood related deaths account for half of
all deaths from natural disasters~\cite{ohlFloodingHumanHealth2000}.  Although
flooding impacts both developed and developing countries, developing nations
face much worse consequences as a result of flooding since they lack resources
to adequately mitigate
disasters~\cite{quarantelliUrbanVulnerabilityDisasters2003}.  Unregulated
urbanization, rising population and climate change are all contributing to
increase the rate at which floods occur in developing megacities; furthermore,
there is little data about these
disasters~\cite{meierDigitalHumanitariansHow2015}.
Data scarcity makes it hard to pinpoint where to direct aid during disasters and
where to make infrastructure improvements after
disasters~\cite{ranaMultidimensionalModelVulnerability2018}.

Various stakeholders have different but overlapping interests with regards to
disaster management.  Government and NGOs work together to provide relief from
mitigate damages from flooding~\cite{chanResilientFloodRisk2018}. Citizens look
for relevant flood information and try to reduce their
risk~\cite{viewegMicrobloggingTwoNatural2010}.  Information is at the core of
this interaction; however, data scarcity makes it hard for emergency personnel
to optimize their use of resources, while citizens have an abundance of
information about their surroundings but must be careful not to trust incorrect
or outdated information about broader
areas~\cite{quarantelliProblematicalAspectsInformation1997}. The natural
solution is for citizens on social media to submit real time reports to the
Emergency Operations Center (EOC), which is tasked with using those reports to
inform citizens.  There is one problem with this solution, in times of crisis
EOCs can suffer from information overload when they are presented with too much
information~\cite{tierneyFacingUnexpectedDisaster2001}.

The REACT system uses novel machine learning and human computer interaction
research to reduce information overload in EOCs, thereby decreasing disaster
response time. REACT learns how Emergency Operations Centers (EOCs) classify the
severity of flood events given citizen submitted reports. REACT trains itself
through a gamified simulation of a disaster event. During a real disaster, REACT
digests social media reports and estimates how severely an event is impacting
different areas of a city and thereby helps EOCs to respond in the best manner
possible.

\section{Emerging Risk In South and Southeast Asia}
Global climate change is `expected to increase the frequency and intensity of
floods'~\cite{ahernGlobalHealthImpacts2005}.  Urban areas are particularly at
risk from flooding; unchecked development and rising population have created
megacities that regularly experience flooding~\cite{chanFloodRiskAsia2012}. Nowhere is
this more apparent than in South and Southeast Asia, where the severity of
floods has been increasing over the past several
decades~\cite{tortiFloodsSoutheastAsia2012}.

Of the world's 33 megacities, over 60 percent are located in developing Asian
Countries~\cite{unitednationsdepartmentofeconomicandsocialaffairsWorldCities20162016}.
These cities face a looming crisis as flood risk increases, but there are also
unique opportunities for risk mitigation. Megacities are characterized by high
population density. This high density leads to an increase in economic damages
and loss of life, but it also means that there are large numbers of citizens
that have disaster information they'd like to share with
others~\cite{chanFloodRiskAsia2012}.

\section{History of Disaster Informatics}

Work in Mapping disasters Epidemiology
John Snow's use of maps to find the source of Cholera outbreak in
London\cite{rogersJohnSnowData2013}.


In more recent times, the need for Information Technology (IT) in disaster
management has been clear since the mid 1980s when computers became user
friendly enough to be used during
disasters~\cite{universityTerminalDisastersComputer1986}. Now many EOCs use
Geographic Information Systems (GIS) in order to organize spatial data and
analyze disaster information; however, researchers have often stated that `there
are many reasons to remain skeptical about the idea that technology will provide
a panacea for emergency management problems'~\cite{tzemosUseGISFederal1995,
tierneyFacingUnexpectedDisaster2001, perryNaturalDisasterManagement2007}.
A number of potential negative effects associated with
IT disaster management technology have been identified: the potential for the
technology to increase social inequality, the potential of information overload,
and the dissemination of incorrect and outdated information (Quarantelli 1997;
Flentge et al., n.d.).


For flooding:~\cite{ahernGlobalHealthImpacts2005}

Technology can help disaster response; however, it also has the ability to cause
information overload\cite{tierneyFacingUnexpectedDisaster2001}


\subsection{Social Media and disasters} The history of social media and the
hashtag is invariably linked to disaster communication.  It was during the San
Diego bush fires of 2007 that hashtags were first widely used on
twitter~\cite{salazarHashtagsAnnotatedHistory2017}.


Much work has been done in passively listening to social media streams in order
to better understand how disasters unfold and how humans use social media as a
communication tool during disaster events. Many of these studies use hand
labeled tweets in order to analyze which percentage of them 

Digital Humanitarians use twitter to help spatially locate needs in Haiti after
the 2008 earthquake~\cite{meierDigitalHumanitariansHow2015}

Twitter tale of 3 hurricanes~\cite{alamTwitterTaleThree2018}

\subsection{Machine Learning and Social Media} 



\section{The Riskmap System}\label{ch1:riskmap}


\subsection{Motivation for crowdsourced data} Citizens as sensors Geosocial
intelligence
Holderness~\cite{holdernessSocialMediaGeoSocial2015a}~\ref{ch1:riskmap}.

Quarantelli emphaszied that  `management of hazards is fundamentally social in
nature and not something that can be achieved strictly through technological
upgrading'~\cite{tierneyFacingUnexpectedDisaster2001} yet social media brings
human behavior into a machine readable format that can be used to provide
further information during disasters.


\subsection{}

