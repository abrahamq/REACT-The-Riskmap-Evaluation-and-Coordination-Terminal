\chapter{Previous Work}

\section{Machine Learning in Crisis Informatics}

\subsection{Passive Listening}

Some was looking for clues after disasters:
\cite{viewegMicrobloggingTwoNatural2010}
problem: don't have real time results

Some used humans to filter out social media for real time disaster info
\cite{starbirdVoluntweetersSelforganizingDigital}
\cite{meierDigitalHumanitariansHow2015}

Some involved using machine learning:
\cite{imranPracticalExtractionDisasterrelevant2013}
\subsubsection{Problems with that approach:}
It is very difficult to filter out which social media images are related to the
disaster and which are not.


\subsection{On Image Data}
Online learning using traditional transfer
learning~\cite{donahueDeCAFDeepConvolutional2013} but online (so they get people
to label images as a disaster happens?) plus train with generic disaster
images in order to solve cold start problem at the beginning of an event.
Classify social media images into 3 classes: (severe, mild, little) damage.
\cite{nguyenDamageAssessmentSocial2017}



\subsection{On Text Data}\label{chap3:text}
Crowd sentiment detection during disasters using twitter and the 2010 San Bruno
CA fires n=3698
\cite{nagyCrowdSentimentDetection2012}

Feature engineering on twitter messages to classify into pre-incident, during
incident and post-incident
\cite{chowdhuryTweet4actUsingIncidentspecific2013}

Classifying tweets as informative/ not informative using CNNs vs SVMs (CNN wins)
\cite{carageaIdentifyingInformativeMessages2016}

CrisisNLP from the Qatar Computing Research Institute
has a huge datasets
\cite{nguyenRapidClassificationCrisisRelated}


\subsection{Ensemble Data Models}
Boosting of Tree-Based Classifiers for Predictive Risk Modeling in GIS
\cite{furlanelloBoostingTreeBasedClassifiers2000}

This paper~\cite{mouzannarDamageIdentificationSocial2018} uses deep learning to
identify damage related info 

Low level visual features (extract color, shape texture) + then Use bag of words
on the text. Make a 
\cite{jomaaSemanticVisualCues2016}

There have been some notable projects that attempt to provide complete systems
that can be used for different disasters. Most notably are the Sahana and the
AIDR projects.

Sahana has suspended its disaster response project that helped to
mobilize volunteers to respond to disasters.
// not focusing enough on the HCI and hidden wiring?


\subsection{Common Difficulties}
\subsubsection{Task Subjectivity}
Task subjectivity is an incredibly common
issue~\cite{nguyenDamageAssessmentSocial2017,
quarantelliUrbanVulnerabilityDisasters2003}. While most humans can agree on
whether an object is or is not an apple, this task does not translate to
defining if a picture indicates a severe event or a minor one. 

In other words, people's perception of risk varies widely from region to region
and from citizen to citizen~\cite{quarantelliUrbanVulnerabilityDisasters2003}.

\subsubsection{Small Datasets}
Although larger datasets have recently become available, there has
historically been a scarcity of training and validation data available for
Deep learning models that are trained on small datasets tend to overfit on the
training data and do not generalize well to the validation
dataset~\cite{perezEffectivenessDataAugmentation2017}.

In many early studies only hundreds of data points were considered--- combined
with the small size of those data points (for example, twitter microblogs of
140 or 280 characters) and effectively using deep learning becomes very
difficult.
% TODO for Adi should I not write this? % 
For example~\cite{nagyCrowdSentimentDetection2012} only uses 3,698 tweets in
order to train 

\subsubsection{Connects citizens to EOC}
As discussed in~\ref{chap1:riskmap}, the Riskmap system helps to connect
citizens to Emergency Operations Centers

\subsubsection{Focus on technology rather than whole system design}
A series of UN case studies on six disaster information systems found that while
engineering and system design were essential, it was the hidden wiring of support
networks that allows for technology to succeed.

\begin{quote}
An important message emerges from the case studies: an effective disaster
information management system requires a good technological platform,
but also much more. Software programs for storing, sharing, and manipulating
data for disasters are being developed or patched together at a steady pace,
often in the aftermath of disasters. The real difficulty lies in anchoring
these technological approaches in an appropriate institutional context where
they are supported by relevant and effective operating procedures, agreed
terminology and data labeling, and a shared awareness of the benefits of proper
handling of disaster information. Clearly, a disaster information management
system must be supported by accepted rules, procedures, and relationships
that encourage, facilitate, and guide the production, sharing, and analysis and
use of data in response to disaster. In these case studies, the institutional
dimension---the hidden wiring---determined the effectiveness of the
systems.~\cite{aminDataNaturalDisasters2008}
\end{quote}

