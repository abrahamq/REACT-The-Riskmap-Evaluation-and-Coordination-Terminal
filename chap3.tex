\chapter{Previous Work for Machine Learning in Crisis Informatics}

As discussed in Section~\ref{chap2:socialMedia}, researchers have enviously eyed
the panacea of information present in social media streams since the inception
of the medium. Social media datasets are often very large, often hundreds of
thousands or millions of data points can be collected during a single event. For
example more than 1.5 million disaster related tweets were created during the
2011 Tohoku earthquake~\cite{doanAnalysisTwitterMessages2012} and over 20
million during hurricane Sandy~\cite{meierDigitalHumanitariansHow2015}. While
all of these tweets concern the natural disaster and citizen's reaction to the
event, only a few have actionable intelligence that can help Emergency
Operations Systems to better respond to an event--- filtering out these 
actionable tweets and thus better understanding the ground truth is at the core 
of much research. 

\section{Passive Collection after Event}
Because of the large number of data points and because of the difficulty of
using humans to label all of the data set, researchers have turned to keyword
searches to whittle down the volume of
data~\cite{doanAnalysisTwitterMessages2012}. Researchers then either hand label
a small subset of the data or use keywords to aggregate data points and examine
statistics on those aggregates.

\subsection{keyword searches and hand labeling}
Many studies of social media disaster data focused around post-event analysis
using manual labeling. In~\cite{starbirdVoluntweetersSelforganizingDigital}
Starbird and Palen explore the uptake of the Tweek the Tweet (TtT) microsyntax, where
citizens are encouraged to use a specific syntax in order to request or offer
help during a crisis event. The adoption of the syntax by citizens on the ground
was very low. Only 39 out of a total of 2,911 tweets in the TtT dataset originated within the affected
areas; however, the authors saw many more tweets that were translated into the
TtT syntax by other users. The authors also point out that social media enable new
forms of volunteerism where humans can be used to manually `translate' user tweets.

Other studies have analyzed social media behavior by using key word searches only and
creating aggregate statistics. For example, Vieweg et al.\ state that they `reduced
the data sets to those user streams that included more than three tweets
containing the search terms' in order to 
`make samples manageable'~\cite{viewegMicrobloggingTwoNatural2010}. The team then looked
at easily machine readable features such as the inclusion of geotags.

\section{Artificial Intelligence}
While it is possible to gather some intelligence by using key words and
digital volunteers to label data, artificial intelligence has two main advantages. 
Machine learning algorithms can handle much larger volumes of data than can be 
consumed by even a large team of people, and they can do it in a much 
shorter amount of time.

\subsection{On Text Data}\label{chap3:text}
Nagy et al.\ used SentiWordNet, a lexical embedding for sentiment analysis, in order 
to classify tweets as mainly containing positive or negative feelings~\cite{nagyCrowdSentimentDetection2012}.
A dataset of 3698 tweets created during the 2010 San Bruno, CA fires was used in this study. 
The authors did manually label the tweets in order to find the performance of their
method and found a precision score of 90\% on a held out set. After validation,
the authors can use this process to classify tweets with little human interaction,
allowing them to process much larger datasets.

In 2013, Chowdhury et al.\ used feature extraction to classify tweets into
pre-incident, during-incident, and post-incident categories~\cite{chowdhuryTweet4actUsingIncidentspecific2013}.
They showed the importance of treating crisis messages as distinct from other
text based machine classification tasks. The precision score of this method 
was at least 96\%~\cite{chowdhuryTweet4actUsingIncidentspecific2013}.

Caragea et al.\ and Imran et al.\ compared the performance of Support Vector Machines (SVMs) and Convolutional 
Neural Networks (CNNs) at classifying tweets into two classes: informative, and not
informative~\cite{carageaIdentifyingInformativeMessages2016,imranPracticalExtractionDisasterrelevant2013}. The
teams found that CNNs performed much better at the classification task because they were better 
able to learn the relationships between tokens and did not require much feature engineering in order 
to have good performance.

In addition to methods for classifying disaster messages, there has recently been work 
on creating textual datasets to help researchers conduct machine learning research. 
The Qatar Computing Research Institute has created the CrisisNLP datasets that 
contain thousands of tweets that have been human labeled~\cite{nguyenRapidClassificationCrisisRelated}.

\subsection{On Image Data}\label{chap3:image}
In the past decade there have been very promising results to using Convolutional Neural Networks
for image recognition tasks, however training these large neural nets traditionally
requires very large datasets. In 2014, the DeCAF team showed how to conduct 
transfer learning on deep neural nets in order to reduce the number of 
training images required without sacrificing performance~\cite{donahueDeCAFDeepConvolutional2013}.
Following this result, Nguyen et al.\ used transfer learning with an online component 
to classify social media disaster images into 3 classes: severe, mild, and little
damage~\cite{nguyenDamageAssessmentSocial2017}. 

  visual bag of words, then Low level visual features (extract color, shape
  texture) + then use bag of words on the text. Make
  a~\cite{jomaaSemanticVisualCues2016}

\subsection{Ensemble Learning Models}
  The prospect of using many learning models and combine them has driven many 
  research goals. Bagging, the practice of using many simple learners on
  different subsets of the training data has shown to be particularly
  useful~\cite{breimanBaggingPredictors1996}.  Boosting is a related technique
  where an emsemble of simple classfiers are trained in succession, with
  data points misclassified by the nth classifer are weighed as more important
  for classifier $n+1$. In a disaster data context, Furlanello et al.\ used
  boosting to train decision trees and geospatially predict the risk of
  tickborne illnesses~\cite{furlanelloBoostingTreeBasedClassifiers2000}.
  
  
  \subsubsection{Multimodal data} 
  Ensemble models are not only attractive because they can increase precision on a
  single type of data, but also because they can enable classifiers to process
  many types of data. For example, Mouzannar et al.\ used two different CNNs to 
  process image and text data. The team then used a variety of simple classifiers 
  as the ensemble learners to produce a final result,
  classification into one of 6 classes~\cite{mouzannarDamageIdentificationSocial2018}.
  In this manner, they were able to achieve higher than 90\% accuracy with a dataset of 
  35,000 data points.

  Researchers have also used neural networks as the ensemble learners. These models enable
  much more flexibility because the learner is able to encode a much more complex
  function and can therefore learn more information about the
  latent features between learners~\cite{jordanHierarchicalMixturesExperts1994}.

\section{Challenges}

\subsection{Small Datasets}
Although a limited number of larger datasets have recently become available, there has
historically been a scarcity of training and validation data available for
Deep learning models that are trained on small datasets tend to overfit on the
training data and do not generalize well to the validation
dataset~\cite{perezEffectivenessDataAugmentation2017,nguyenRapidClassificationCrisisRelated}.

In many early studies only hundreds of data points were considered--- combined
with the small size of those data points (for example, twitter microblogs of
140 or 280 characters) and effectively using deep learning becomes very
difficult. For example~\cite{nagyCrowdSentimentDetection2012} only uses 3,698 tweets in
order to train their classifier.
\subsection{Cold Start}

At the beginning of a disaster, there is not much data available in order to
train a classifier.

\subsection{Task Subjectivity}
Task subjectivity is an incredibly common
issue~\cite{nguyenDamageAssessmentSocial2017,
quarantelliUrbanVulnerabilityDisasters2003}. While most humans can agree on
whether an object is or is not an apple, this task does not translate to
defining if a picture indicates a severe event or a minor one. 
In other words, people's perception of risk varies widely from region to region
and from citizen to citizen~\cite{quarantelliUrbanVulnerabilityDisasters2003}.

Houses look different in different built environments, so cross location
transfer might not work as well as we think it will.


\subsubsection{Focus on technology rather than whole system design}
A series of UN case studies on six disaster information systems found that while
engineering and system design were essential, it was the hidden wiring of support
networks that allows for technology to succeed.

There have been some notable projects that attempt to provide complete systems
that can be used for different disasters. Most notably are the Sahana and the
AIDR projects.

Sahana has suspended its disaster response project that helped to
mobilize volunteers to respond to disasters.
\cite{SahanaAGM2018}

  
\begin{quote}
An important message emerges from the case studies: an effective disaster
information management system requires a good technological platform,
but also much more. Software programs for storing, sharing, and manipulating
data for disasters are being developed or patched together at a steady pace,
often in the aftermath of disasters. The real difficulty lies in anchoring
these technological approaches in an appropriate institutional context where
they are supported by relevant and effective operating procedures, agreed
terminology and data labeling, and a shared awareness of the benefits of proper
handling of disaster information. Clearly, a disaster information management
system must be supported by accepted rules, procedures, and relationships
that encourage, facilitate, and guide the production, sharing, and analysis and
use of data in response to disaster. In these case studies, the institutional
dimension---the hidden wiring---determined the effectiveness of the
systems.~\cite{aminDataNaturalDisasters2008}
\end{quote}

