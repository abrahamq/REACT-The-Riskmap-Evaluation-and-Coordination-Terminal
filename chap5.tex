\chapter{Future Work}

\section{Image Data}
\subsection{Using transfer learning}
As the time goes on and the Riskmap System registers more flood events, 
more data will be collected, thereby increasing the  

\section{Text Data}
\subsubsection{word2vec}
As outlined previously, we did not experiment with word2vec embeddings because
of the difficulty of handling multilanguage datasets; however, it would be
possible to create a word2vec embedder for Indonesian by using publicly
available texts such as Wikipedia. The short coming of using Indonesian Wikipedia 
is that there are only 500 thousand articles compared to 5.9 million in English Wikipedia.

\section{Flood Height}
\section{Location Information}

\section{Ensemble Methods}
\subsection{Bigger network}
Our current bagging network is very small so as to reduce over fitting on the
small datasets we currently have. As the Riskmap System collects more data, it
is likely that we can use a larger network with a decreased risk of over fitting.

\subsection{Data Augmentation}
It might be possible to use generative adversarial learning in order to create
new example reports that fit into the distribution of user submitted reports.
This would increase the size of the dataset without having to deploy the
Riskmap system to more locations. One of the challenges related to
generating new reports would be the multi-language problem and validating
that generated report texts in other languages fit the distribution of user
submitted reports in that language.

% TODO conclusion