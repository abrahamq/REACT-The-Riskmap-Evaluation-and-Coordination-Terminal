\chapter{Future Work and Conclusion}

\section{Future Work} As the time goes on and the Riskmap System registers more
flood events, more training data will be collected. With more training data,
transfer learning will become more and more accurate and the cold start problem
will be less of a burden. It is also possible to use neural network embeddings
for textual report data. We did not experiment with word2vec embeddings because
of the difficulty of handling multilanguage datasets; however, it would be
possible to create a word2vec auto embedder for Indonesian by using publicly
available texts such as Wikipedia. The short coming of using Indonesian
Wikipedia is that there are only 500 thousand articles compared to 5.9 million
in English Wikipedia.
Furthermore, it might be possible to use generative adversarial learning in
order to create new example reports that fit into the distribution of user
submitted reports.  This would increase the size of the dataset without having
to wait for disaster events to occur. One of the challenges related to
generating new reports would be the multi-language problem and validating that
generated report texts in other languages fit the distribution of user submitted
reports in that language.

The current ensemble learner uses a very small dense neural network to reduce
over fitting on small datasets. As the Riskmap System collects more data, it
is likely that a larger network can be used without overfitting, thereby
increasing testing accuracy.

Location information was not used because we aimed to create a per report
heuristic for the whole city rather than concentrate on spatially disparate
areas. In order to use spatial data it would be necessary to compensate for the
sampling bias present in the dataset: richer neighborhoods have much higher
rates of ownership of smart phones, which means citizens in those areas are more
likely to report flooding than poorer areas--- even if flooding is occurring at
the same levels in both neighborhoods.

\section{Conclusion}
As global climate change continues to increase the frequency and severity of
disaster events worldwide, it will become more and more important for EOCs to
use all the tools at their disposal to make sense of the large amounts of data
available.
The REACT System has been built and trained so that it can be used as
a detection tool to help EOCs find the reports that are most likely to indicate
`heavy flooding'. The REACT System sets itself apart from other disaster
management systems because it is modular and adaptable to new data streams.
Furthermore, the use of machine learning as a service technologies allows REACT
to be less impacted by the cold start problem at the beginning of disasters and
also means that the system will improve as those services improve. 
