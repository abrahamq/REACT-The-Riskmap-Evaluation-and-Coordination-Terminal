
% $Log: abstract.tex,v $
% Revision 1.1  93/05/14  14:56:25  starflt
% Initial revision
% 
% Revision 1.1  90/05/04  10:41:01  lwvanels
% Initial revision
% 
%
%% The text of your abstract and nothing else (other than comments) goes here.
%% It will be single-spaced and the rest of the text that is supposed to go on
%% the abstract page will be generated by the abstractpage environment.  This
%% file should be \input (not \include 'd) from cover.tex.


The United Nations Office for Disaster Risk Reduction (UNDRR) states that economic losses
due to natural disasters have risen 151 percent in the past 20 years. Of these
disasters, floods are the most common. The Sendai Framework for Disaster Risk Reduction
was created by the UNDRR in order to chart goals for future risk mitigation; among its
seven global targets is increasing the availability of disaster risk information
and assessment systems. Disaster information systems use state of the art techniques such as
remote sensing in order to mitigate damages from natural and man made hazards.

It is common in developed countries utilize networks of advanced sensors and
ahead of time mapping in order to facilitate emergency responses;
however, such systems are not available in developing countries due to cost limitations.
The widespread proliferation of smart phones and social media use in
developing countries means that citizens can be used as sensors by reporting
disaster information online. The Riskmap system was developed by the
Urban Risk Lab at MIT in order to gather citizen report streams. Such citizen
disaster reports have two issues:  a large influx of reports can cause
information overload in emergency operations centers, which makes it difficult to
summarize the situation. Machine learning has previously been used in order to
analyze and simplify information for human consumption. This work seeks to use
novel machine learning techniques to fully utilize crowd-sourced social media reports gathered
using the Riskmap system.

    First we establish the motivation for using citizens as sensors and
analyzing this noisy data using machine learning. We then review different
machine learning techniques that have been used in crisis information systems,
including those that also utilize social media.
Finally a novel ensemble learning model is presented that can accurately predict
large flood events from crowdsourced
data.
