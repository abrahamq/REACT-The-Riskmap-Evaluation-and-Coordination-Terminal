\chapter{Methodology}
The Riskmap system allows citizens to easily submit
disaster reports as such it has allowed the Urban Risk Lab at
MIT to gather thousands of reports of real flooding in Indonesia and India.
These data points include requests for help, traffic reports, indications that an
area is safe, and advice for other citizens in the area. Images attached to
reports include a wide variety of scenes, from daylight highways with cars and
motorcycles to night time deserted alleys. Additionally, citizens were asked to
provide an estimated flood height using a slider.

\section{Text}
As shown in Section~\ref{chap1:riskmap}, the Riskmap system allows citizens to provide
a textual description to emergency managers. In Indonesia, most of the reports
are provided in Bahasa, the local language; however, in Chennai all reports were
submitted in English even though the system also supports Tamil. In both Chennai
and Jakarta these reports are quite brief, with the longest reports having 140
characters. In this manner, they are quite similar to tweets which were
initially 140 characters but this limit was doubled in 2017. Helpfully, this
means that much of the work described in~\ref{chap3:text} applies to the Riskmap
text corpus.

Table~\ref{table:text_sample} contains a sample of ten reports that are
indicative of those found in the Riskmap textual descriptions.

\label{table:text_sample}
\begin{table}
  \begin{tabular}{ll}
    \toprule
    {} &
    text \\
    pkey &
    \\
    \hline{}
    \midrule
    169  &
    Waterlogging near cathedral road flyover  \\
    171  &
    1st street Engineers avenue \\
    173  &
    Not that much water safe only \\
    174  &
    50cm water stagnant on the road \\
    176  &
    Water level rising  slowly  \\
    177  &
    Water logging  \\
    178  &
    Model school road is completely flooded, with water almost knee deep \\
    179  &
    Heavy rain in West mambalam flood \\
    180  &
    Water on roads. Stay safe \\
    182  &                                                           4cm
    rainfall.. still continuing.. hope for safe .. dont come outside in
    night time \\
    181  &
    Luz signal flooded knee deep water \\
    \bottomrule
  \end{tabular}
\end{table}

\subsection{Preprocessing}
We first remove test reports, which are reports submitted in order to ensure
that the system is working. The following POSTGRESQL query was executed to remove
reports that were only used to test the system:

\begin{lstlisting}[language=SQL]
SELECT   pkey,
         text
FROM     riskmap.all_reports
WHERE    text IS NOT NULL
AND      Length (text) > 0
AND      text NOT similar TO '%%(T|t)(E|e)(S|s)(T|t)%%'
ORDER BY created_at;
\end{lstlisting}

We then use python to remove punctuation and then split along whitespace, thus
splitting the source text into individual words without any spaces.

\begin{lstlisting}[language=python]
def prepare_text(report_text):
    '''
    returns a list of strings where each string is a different word
    '''
    import string
    exclude = set(string.punctuation)
    s = "".join(ch for ch in inp if ch not in exclude )
    return s.lower().split()
\end{lstlisting}

\subsection{Sentiment analysis}
Since each report in the Chennai dataset includes a textual description in English,
we could use off the shelf sentiment analysis to gauge how
negatively citizens are feeling. It might be the case that a highly negative
sentiment corresponds to heavy flooding and that a positive sentiment
corresponds to lighter or no flooding.  We can investigate the relation between
a negative sentiment and heavy flooding by using conditional probability. We set
the threshhold for negative sentiment at~.5 and then use the AWS Rekognition API
in order to classify texts into heavy flooding when negative sentiment is
greater than $.5$ and into light or no flooding otherwise.

We use bayes' rule in order to analyze the true positive rate--- the
probability that a report represents heavy flooding given a negative sentiment:
\\
$$P( Heavy Flooding | negative) =  \frac{P(heavy Flooding | negative)}{P(negative)} =.65 $$
\\

The false positive rate, which is the probability that there is no heavy
flooding given a negative sentiment is given by:

\\
$$P( No Flooding | negative) =  \frac{P(No Flooding | negative)}{P(negative)} =.34 $$
\\

These probabilities show that while there is some relation between a negative
sentiment and flooding, it is not a very strong signal. Furthermore, there are
no off the shelf sentiment analysis tools for the Indonesian language, so a
model based on AWS Rekogniton or Google Cloud Natural Language API would not
translate to the Jakarta dataset.

\subsection{Bag Of Words}
While sentiment analysis might not be a strong enough signal of heavy flooding/
no heavy flooding, our experiment shows that the textual data contains important
information. In order to train a machine learning algorithm on textual data one
must first create an embedding that maps natural language into feature vectors.
There are many ways of creating embeddings as discussed in
Section~\ref{chap3:text}, but many of them require large datasets or do not
support Indonesian. For example, word2vec is a popular embedding model that
produces floating point vectors and has achieved remarkable performance;
however, the size of its training vocabulary is 962,000 unique
words~\cite{mikolovDistributedRepresentationsWords2013}. It is possible to
download a pre-trained word2vec model and use it to encode new texts, but such a
pre-trained model doesn't exist for Indonesian. We could train it using a
different dataset of Indonesian texts, but there is no guarantee that our domain
specific words would have a good embedding after having trained with a different
corpus.

The bag of words encoding is particularly attractive to the Riskmap
dataset because it language agnostic and can therefore work on both the Chennai
and Jakarta datasets.  The bag of words approach to classifying texts consists
of first creating a vocabulary that maps from a token t to a unique index i. Each
report text is then encoded into a feature vector by setting the ith element to
1 if the token t exists in the report
text~\cite{khuranaNaturalLanguageProcessing2017}.

The bag of words model correctly classifies 66 percent of reports in the Chennai corpus
under 5 fold cross validation. Examining the data, we see that there are many
instances of reports such as `no flooding here' which are being
misclassified because the embedding is not able to understand relationships
between adjacent words.

\section{Images}

\subsection{Visual Bag of Words}

~\cite{yangEvaluatingBagofvisualwordsRepresentations2007}

\section{Flood Height}
