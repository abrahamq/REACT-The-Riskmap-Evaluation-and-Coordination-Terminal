\chapter{Methodology}
\section{Data Description}
The Riskmap system allows citizens to easily submit
disaster reports~\ref{chap1:riskmap}; as such it has allowed the Urban Risk Lab at
MIT to gather thousands of reports of real flooding in Indonesia and India.

\subsection{Image Data}

\subsection{Text Data}
\label{table:text_sample}
\begin{tabular}{l|l}
	\toprule
	{} &
	text \\
	pkey &
	\\
	\hline
	\midrule
	169  &
	Waterlogging near cathedral road flyover  \\
	171  &
	1st street Engineers avenue \\
	173  &
	Not that much water safe only \\
	174  &
	50cm water stagnant on the road \\
	176  &
	Water level rising  slowly  \\
	177  &
	Water logging  \\
	178  &
	Model school road is completely flooded, with water almost knee deep \\
	179  &
	Heavy rain in West mambalam flood \\
	180  &
	Water on roads. Stay safe \\
	182  &                                                           4cm
	rainfall.. still continuing.. hope for safe .. dont come outside in
	night time \\
	181  &
	Luz signal flooded knee deep water \\
	\bottomrule
\end{tabular}

Table \ref{table:text_sample} contains a sample of ten reports that are 
indicative of those found in the Riskmap textual descriptions.

\subsubsection{Preprocessing}
Remove punctuation replaced with whitespace and then split along whitespace.
\subsection{Flood Height}
\subsection{Location Information}

\section{Image Recognition}
\section{NLP}

	\subsection{Bag of Words}
	\subsubsection{Performance}
	67\% on 5 fold cross validation on Chennai dataset with 328 reports. All
	punctuation removed.
		
	\subsection{Bigrams}
	\subsubsection{Performance}
	67\% with bigram model

\section{Flood Height}
