%% This is an example first chapter.  You should put chapter/appendix that you
%% write into a separate file, and add a line \include{yourfilename} to
%% main.tex, where `yourfilename.tex' is the name of the chapter/appendix file.
%% You can process specific files by typing their names in at the 
%% \files=
%% prompt when you run the file main.tex through LaTeX.
\chapter{Previous Work} 

\section{Machine Learning in Crisis Informatics}

\subsection{Passive Listening}
Most of the work in this area has been done by passively listening to twitter
posts or facebook comments. 

Some was looking for clues after disasters: 
\cite{viewegMicrobloggingTwoNatural2010}
problem: don't have real time results

Some used humans to filter out social media for real time disaster info
\cite{starbirdVoluntweetersSelforganizingDigital}
\cite{meierDigitalHumanitariansHow2015}

Some involved using machine learning:
\cite{imranPracticalExtractionDisasterrelevant2013}

\subsubsection{Problems with that approach:}
It is very difficult to filter out which social media images are related to the
disaster and which are not.

\subsection{On Image Data}
Online learning using traditional transfer
learning~\cite{donahueDeCAFDeepConvolutional2013} but online (so they get people
to label images as a disaster happens?) plus train with generic disaster
images in order to solve cold start problem at the beginning of an event.
Classify social media images into 3 classes: (severe, mild, little) damage.
\cite{nguyenDamageAssessmentSocial2017}



\subsection{On Text Data}
Crowd sentiment detection during disasters using twitter and the 2010 San Bruno
CA fires n=3698
\cite{nagyCrowdSentimentDetection2012}

Feature engineering on twitter messages to classify into pre-incident, during
incident and post-incident
\cite{chowdhuryTweet4actUsingIncidentspecific2013}


Classifying tweets as informative/ not informative using CNNs vs SVMs (CNN wins)
\cite{carageaIdentifyingInformativeMessages2016}



CrisisNLP from the Qatar Computing Research Institute
has a huge datasets 
\cite{nguyenRapidClassificationCrisisRelated}





\subsection{Ensemble Data Models}

\subsection{Common Difficulties}
\subsubsection{Task Subjectivity}
Task subjectivity is an incredibly common
issue~\cite{nguyenDamageAssessmentSocial2017, quarantelliUrbanVulnerabilityDisasters2003}. While most humans can agree on
whether an object is or is not an apple, this task does not translate to
defining if a picture indicates a severe event or a minor one. 

In other words, people's perception of risk varies widely from region to region
and from citizen to citizen~\cite{quarantelliUrbanVulnerabilityDisasters2003}.

\subsubsection{Small datasets}
For example~\cite{nagyCrowdSentimentDetection2012} only uses 3,698 tweets.
