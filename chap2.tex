\chapter{Background}

%% TODO I feel like I'm restating myself here?
Global climate change is `expected to increase the frequency and intensity of
floods'~\cite{ahernGlobalHealthImpacts2005}. Developing urban areas that are
undergoing unchecked development and rising population regularly experience
flooding~\cite{chanFloodRiskAsia2012}.  Nowhere is this more apparent than
in South and Southeast Asia, where the severity of floods has been increasing
over the past several decades~\cite{tortiFloodsSoutheastAsia2012}.

Of the world's 33 mega-cities with population over 10 million, more than 60
percent are located in developing Asian
Countries~\cite{unitednationsdepartmentofeconomicandsocialaffairsWorldCities20162016}.
These cities face a looming crisis as flood risk increases, but there are also
unique opportunities for risk mitigation. Megacities are characterized by high
population density, which leads to an increase in economic damages and loss of
life during flood events, but it also means that there are large numbers of
citizens that have disaster information they would like to share with
others~\cite{chanFloodRiskAsia2012}.

\section{History of Disaster Informatics} One of the best known and earliest
work in disaster informatics was John Snow's use of maps to find the source of
the 1857 Cholera outbreak in London\cite{rogersJohnSnowData2013}. This example
is taught to all students of epidemiology and illustrates the need
not only for up to date information, but for systems that ease the
analysis of this information. In John Snow's case, the map was the technology
that allowed him to visualize the spread of the disease and effectively take
action that ended the outbreak; however, the computer revolution has drastically
changed the way that scientists and responders analyze disaster information.
John Snow used mapping technology to track the spread of disease, but now
researchers are using artificial intelligence to predict cholera outbreaks
before they happen~\cite{radinskyMiningWebPredict2013}.

%% this needs to get moved somewhere else, but I'm not sure where...
In more recent times, the need for Information Technology (IT) in disaster
management has been clear since the mid 1980s when computers became user
friendly enough to be used during
disasters~\cite{universityTerminalDisastersComputer1986}.

Now many Emergency Operations Centers (EOCs) use Geographic Information Systems
(GIS), inventory control systems, and online messaging systems among other
technology in order to organize spatial data and analyze disaster information.
For example, Mozambique used an integrated disaster management system to provide
early warning during the 2007  Zambezi floods, while Guatemala's inventory
management helped to curb government bribes~\cite{aminDataNaturalDisasters2008}.
While technology has helped some EOCs to better respond to disaster events,
researchers have often stated that `there are many reasons to remain skeptical
about the idea that technology will provide a panacea for emergency management
problems'~\cite{tzemosUseGISFederal1995, tierneyFacingUnexpectedDisaster2001,
perryNaturalDisasterManagement2007}. A number of potential negative effects
associated with disaster management technology have been identified: primarily
the potential of information overload and the dissemination of incorrect and
outdated information~\cite{quarantelliProblematicalAspectsInformation1997,
flentgeDesigningContextAwareHCI}.

\subsection{Social Media and Disasters}\label{chap2:socialMedia}
%% TODO is this in the right place??
The history of online communities is firmly linked to disasters. Internet
Relay Chat (IRC) was one of the first truly global online communication systems;
its adoption among internet connected citizens was `prompted by the First Gulf
War'~\cite{salazarHashtagsAnnotatedHistory2017}. Although radio and
television broadcasts were halted by the Iraqi army shortly after the invasion,
IRC communication continued for days afterward. IRC allowed users to communicate
about conditions on the ground, including the Gulf War oil spill that grew to be
the largest oil spill in history~\cite{Timeline20Years2010}.

The history of social media and the hashtag is invariably linked to disaster
communication.  It was during the San Diego bush fires of 2007 that the hashtag
was first widely used on twitter~\cite{salazarHashtagsAnnotatedHistory2017}.

Quarantelli, a pioneer of disaster sociology, emphasized that  `management of
hazards is fundamentally social in nature and not something that can be achieved
strictly through technological
upgrading'~\cite{tierneyFacingUnexpectedDisaster2001} yet social media brings
human behavior into a machine readable format that can provide
information that was previously hidden.

Work has been done in passively listening to social media streams in order
to better understand how disasters unfold and how humans use social media as a
communication tool during disaster events. Many of these studies use hand
labeled tweets in order to classify what kind of information people talk
about~\cite{alamTwitterTaleThree2018}.Further work has evolved to using
artificial intelligence methods to automatically label new tweets using
supervised learning. For example, Patrick Meier's Haiti Crisis Map initially
used volunteers to classify large number of tweets, but his more recent projects
focus on the use of AI for tackling big data
problems~\cite{meierDigitalHumanitariansHow2015}.

\subsection{Crowdsourcing vs. Passive Listening} 
As Patrick Meier points out in \textit{Digital Humanitarians}, `since
humanitarian organizations don't ask eyewitnesses on social media to report
information', they must passively wait and `rely on witnesses sharing relevant
information by chance'~\cite{meierDigitalHumanitariansHow2015}. Listening to
twitter data streams and hoping that someone posts relevant disaster information
is not always a winning strategy. Past solutions have paid
workers to collect information and enter it into disaster information
systems. The government of Mozambique used this technique during flooding in
2007; however, this method was found to be expensive and
unscaleable~\cite{aminDataNaturalDisasters2008}. An analysis of the system found
that `data processing and consolidation [were] difficult' and that `the few data
entry clerks struggled to keep up'~\cite{aminDataNaturalDisasters2008}.

%% TODO not sure where to place this section?
\section{The Riskmap System}\label{chap1:riskmap}
  \begin{figure}
    \includegraphics[width=\linewidth]{riskmap.png}
    \caption{Submitting a flood report card. Graphic by MIT-URL}\label{fig:cards}
  \end{figure}
  The Riskmap System was created by the Urban Risk Lab (MIT-URL) in 2016 to
  collect citizen reports on disaster events. It alleviates the load on
  emergency managers by centralizing reports from many social media
  sources. Riskmap not only makes it easy for reports to come into the
  response center, but it also allows emergency managers to indicate which areas
  of a city are most affected at any one time. The data gathered during an
  event is persistent and available under the Creative Commons license,
  which allows researchers to track the flood over time and pinpoint areas that
  are particularly vulnerable to flooding, thus fulfilling the need
  for open data~\cite{PhilippinesPDCCollaborate,
  antaranews.comBNPBPetaBencanaId}.

  The system has been in place in Jakarta and Chennai since 2016, and has
  seen hundreds of thousands of views during flood
  events~\cite{noveckOpinionElectionsWon2018, oct31ChennaiGetsRain}.

  \subsection{Social Media Outreach}
  Riskmap consists of many different social media bots that are actively
  filtering social media streams and looking for citizens that might be
  reporting flooding events, it then reaches out to those users and asks
  them to submit a flood report that consists of a GPS location, the
  estimated flood height at that location, a picture, and a textual
  description. The user interface for submitting these reports is shown
  in \figurename{}~\ref{fig:cards}. These reports are displayed on a public map for other
  citizens to inform themselves. Furthermore, EOC personnel are able to
  access the Risk Evaluation Matrix (REM), a special dashboard that allows them
  to give even more information to citizens.

  \subsection{Need for open data} Creating bespoke information systems at
  the beginning of disasters has been the
  norm~\cite{aminDataNaturalDisasters2008}; however this means that disaster
  response organizations must become acclimated with the system at the same time
  that they are dealing with disaster situations.  Researchers have shown the
  need to create open sourced crowdsourced emergency systems that provide open
  data~\cite{avvenutiNeedOpeningCrowdsourced2018a}. The Riskmap System was
  created to fill that need.
  
\section{Conquering Information Overload} 
  When researchers have tried using social media to track real-time disasters,
  they often suffer from information overload. For example, Meier states that
  the Haiti Crisis map was `constantly overwhelmed with the vast amount of
  information that needed to be monitored and processed' and that the team never
  `managed to catch up' with the backlog of social media
  activity~\cite{meierDigitalHumanitariansHow2015}.
  
  It is not enough to create an advanced system for consuming citizen reports,
  it is also necessary to ensure that this system does not consume resources
  that are already scarce during a disaster event, for example the time of
  emergency workers~\cite{aminDataNaturalDisasters2008}. Furthermore, it is also
  important to reduce the resources needed to create insights because if
  analyzing data is too difficult, then decision makers will make decisions
  without having fully analyzed the
  data~\cite{quarantelliUrbanVulnerabilityDisasters2003}.
  
  Using computers to automatically make sense of disaster data has long
  been a goal in disaster informatics, but only recently have machine
  learning techniques become advanced enough to be implemented in production
  emergency systems~\cite{meierDigitalHumanitariansHow2015}. Image
  recognition algorithms can provide summaries of objects and scenes found
  in user submitted photos~\cite{nguyenRapidClassificationCrisisRelated,
  donahueDeCAFDeepConvolutional2013}. Natural language processing can
  estimate the probability that a textual document is overall negative or
  positive and thereby give EOCs a shorthand way to summarize thousands of
  reports in short amounts of
  time~\cite{nguyenRapidClassificationCrisisRelated,
  nagyCrowdSentimentDetection2012}. Finally, ensemble learning methods can
  learn relationships between disparate datasets and synthesize a single
  result~\cite{mouzannarDamageIdentificationSocial2018}.
  
  In this work we will experiment with different machine learning techniques
  for image recognition, finally showing that transfer learning at the 
  output layer can turn off the shelf multi---label  classification
  algorithms into classifiers for flood image classification. For textual analysis, 
  we will show the performance of two commonly used techniques to classify
  report texts into `heavy flooding' or `no heavy flooding' classes.  Flood
  height will first be assessed as a raw numerical feature. Finally, the output
  of these disparate techniques will be combined by using a small deep neural
  network to classify a report into one of two classes: `heavy flooding' or `no
  heavy flooding'

  In order to increase interpretability, the most important labels from
  the feature extraction machine learning algorithms are used to explain the
  decisions of the system.
